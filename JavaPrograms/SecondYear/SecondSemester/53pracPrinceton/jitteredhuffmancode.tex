\documentclass[a4wide,11pt]{article}
% Try the next 
%\documentclass[a4wide,11pt]{report}
% You need to insert some chapters for a report, or a thesis using
% \chapter{Title of the Chapter}\label{chapter:aname}
%  
\usepackage{latexsym,graphicx,wrapfig,psfrag,amssymb}
\usepackage[round]{natbib}

\begin{document}
%\twocolumn     % Remove the comment '%' character to get a double column article
\author{
	Name Surname\\
	Department of Computer Science\\
	University of the Western Cape\\
	\vspace{5pt}
	{\small 1234567@myuwc.ac.za}}
\title{The Effects of Jittering Huffman Code}
\maketitle
\begin{abstract}
Put a short summary of your results here.
\end{abstract}
\section{Introduction}
\label{section:introduction}
Say what it is that you are reporting about.  Huffman discovered his algorithm while he 
was a graduate student and published it as a conference paper~\cite{huffman52}.
\section{Experiment}
\label{section:experiment}
Explain the experiment that you ran. Tabulate some of your results here.
\begin{table}[h]
\caption{Size of error and length of unrecovered code after error}
\label{table:recovery}
\begin{center}
\begin{tabular}{rl|p{36pt}|p{36pt}}
\hline
Error in bits & Length of unrecovered code & See how this is typeset & See how this is typeset \\
\hline
            8 & your figures\\
           16 & your figures\\
           32 & your figures\\
           24 & your figures\\
           40 & your figures\\
           48 & your figures\\
           64 & your figures\\
\hline
      Average & average      \\
Standard deviation & std. dev.\\        
\hline
\end{tabular} 
\end{center}
\end{table}
Say something about Table~\ref{table:recovery} on Page~\pageref{table:recovery}.  % Notice how the reference was done.
\section{Conclusion}
\label{section:conclusion}
Say in summary what the results were. Say something like:  

Huffman code appears to be very resilient to noise.  We have found that the
code recovers within a mean of $xxx$ characters, with a standard deviation
of $\sigma = xxx.xx$ characters after encountering an 8-bit error. 
When encountering a 16-bit contiguous error it seems to recover similarly quickly.
The size of the errors do not seem to have any bearing on recovering subsequent code.

Say what happens when 24-, 32-, 40-, 48- and 64-bit errors are made.

% To include tex files simply do
\include{profilehuffman}% .tex must be omitted
\noindent The profile of David Huffman comes from~\cite{stix91}

% Pleasse do not use \nocite{*}.  It causes all the items in the bibliography to 
% be cited.
%\nocite{*}
\bibliographystyle{plainnat}
\bibliography{jitteredhuffmancode}  % This comes from the file: huffmanwithnoise.bib
% To check if your bibliography is correct run the command:
% bibtex huffmanwithnoise
% If this runs without any error messages your bibliography has no errors.
\vfill
\pagebreak
% \tableofcontents   % Not necessary for a small document, or an article
\end{document}
