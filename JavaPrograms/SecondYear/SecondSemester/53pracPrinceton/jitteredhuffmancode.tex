\documentclass[a4wide,11pt]{article}
% Try the next 
%\documentclass[a4wide,11pt]{report}
% You need to insert some chapters for a report, or a thesis using
% \chapter{Title of the Chapter}\label{chapter:aname}
%  
\usepackage{latexsym,graphicx,wrapfig,psfrag,amssymb}
\usepackage[round]{natbib}

\begin{document}
%\twocolumn     % Remove the comment '%' character to get a double column article
\author{
	Name Surname\\
	Department of Computer Science\\
	University of the Western Cape\\
	\vspace{5pt}
	{\small 1234567@myuwc.ac.za}}
\title{The Effects of Jittering Huffman Code}
\maketitle
\begin{abstract}
Put a short summary of your results here.
\end{abstract}
\section{Introduction}
\label{section:introduction}
Say what it is that you are reporting about.  Huffman discovered his algorithm while he 
was a graduate student and published it as a conference paper~\cite{huffman52}.
\section{Experiment}
\label{section:experiment}
Explain the experiment that you ran. Tabulate some of your results here.
\begin{table}[h]
\caption{Size of error and length of unrecovered code after error}
\label{table:recovery}
\begin{center}
\begin{tabular}{rl|p{36pt}|p{36pt}}
\hline
Error in bits & Length of unrecovered code & See how this is typeset & See how this is typeset \\
\hline
            8 & your figures\\
           16 & your figures\\
           32 & your figures\\
           24 & your figures\\
           40 & your figures\\
           48 & your figures\\
           64 & your figures\\
\hline
      Average & average      \\
Standard deviation & std. dev.\\        
\hline
\end{tabular} 
\end{center}
\end{table}
Say something about Table~\ref{table:recovery} on Page~\pageref{table:recovery}.  % Notice how the reference was done.
\section{Conclusion}
\label{section:conclusion}
Say in summary what the results were. Say something like:  

Huffman code appears to be very resilient to noise.  We have found that the
code recovers within a mean of $xxx$ characters, with a standard deviation
of $\sigma = xxx.xx$ characters after encountering an 8-bit error. 
When encountering a 16-bit contiguous error it seems to recover similarly quickly.
The size of the errors do not seem to have any bearing on recovering subsequent code.

Say what happens when 24-, 32-, 40-, 48- and 64-bit errors are made.

% To include tex files simply do
\noindent{\Large\textbf{Scientific American Article}}\\ 
From the September 1991 issue of Scientific American, pp. 54, 58.\\*[+12pt]

\noindent{\Large{PROFILE:} \textsc{%\bffamily 
David Albert Huffman}\\ 
%(August 9, 1925--October 7, 1999) \\*[+12pt]
\textbf{\Large Encoding the ``Neatness'' of Ones and Zeroes}\\*[+12pt]}

Large networks of IBM computers use it. So do high-definition television, 
modems and a popular electronic device that takes the brain work out of 
programming a videocassette recorder. All these digital wonders rely on the 
results of a 40-year-old term paper by a modest Massachusetts Institute of 
Technology graduate student---a data compression scheme known as Huffman encoding.

In 1951 David A. Huffman and his classmates in an electrical engineering 
graduate course on information theory were given the choice of a term paper 
or a final exam. For the term paper, Huffman's professor, Robert M. Fano, 
had assigned what at first appeared to be a simple problem. Students were 
asked to find the most efficient method of representing numbers, letters 
or other symbols using a binary code. Besides being a nimble intellectual 
exercise, finding such a code would enable information to be compressed for 
transmission over a computer network or for storage in a computer's memory.

Huffman worked on the problem for months, developing a number of approaches, 
but none that he could prove to be the most efficient. Finally, he despaired 
of ever reaching a solution and decided to start studying for the final. 
Just as he was throwing his notes in the garbage, the solution came to him. 
``It was the most singular moment of my life,'' Huffman says. ``There was the 
absolute lightning of sudden realization.''

That epiphany added Huffman to the legion of largely anonymous engineers 
whose innovative thinking forms the technical underpinnings for the accoutrements 
of modem living---in his case, from facsimile machines to modems and a myriad 
of other devices. ``Huffman code is one of the fundamental ideas that people 
in computer science and data communications are using all the time,'' says 
Donald E. Knuth of Stanford University, who is the author of the multivolume 
series The Art of Computer Programming.

Huffman says he might never have tried his hand at the problem---much less 
solved it at the age of 25---if he had known that Fano, his professor, and 
Claude E. Shannon, the creator of information theory, had struggled with 
it. ``It was my luck to be there at the right time and also not have my professor 
discourage me by telling me that other good people had struggled with this 
problem,'' he says.

%Picture of David Huffman from Scientific American
\begin{figure}[h]
\begin{center}
%\fbox{   % fbox puts a frame around copy.
\includegraphics[scale=0.5,viewport=3 3 203 291]{figures/davidAlbertHuffman.jpg}% The jpeg file 
%     The option viewport defines the boundaries of the picure.
%               [viewport = swX, swY,  neX, neY]
%                the coordinates are those of the south-western corner (swX, swY)
%                                         and the north-eastern corner (neX, neY). 
%}        % end of fbox.  It is now commented out.
\caption{\textsc{David A. Huffman} expresses mathematical theorems in intricate 
% In a figure the caption is always underneath the figure.
paper sculptures. \hspace{+150pt} Photo: Matthew Mulbry}
\label{figure:huffmanpic}% Put the label of the figure immediately after its caption.
% You can refer to the figure as follows:
% "... Huffman's picture is shown in Figure~\ref{figure:huffmanpic}."
% This is rendered as: 
% "... Huffman's picture is shown in Figure 12."
% When editing a document, you do not have to keep track of figure numbers.
% Use a capital letter when referring to figure as above.
%
\end{center}
\end{figure}


Like many codes, including the one named after Samuel Morse, Huffman's creation 
tried to find a way to assign the shortest codes to those characters used 
most, the longest codes being reserved for those used rarely if at all. This 
process was carried out by forming a so-called coding tree, in which the 
probability that a number, letter or another character will occur is designated as a leaf on a tree.

The two lowest probabilities are summed to form a new probability. Combining 
of probabilities continues along the branches of the tree until the last 
two numbers add up to 1.0, which forms the tree root. Each probability is 
a leaf, and each branch of the tree is assigned a zero or a one. Code words 
are formed by moving along the branches from the root to the top of the tree, 
aggregating the binary digits along the way.

If letters are to be encoded, an E, which might have a probability of 0.13, 
could be represented by the code 101. The three-digit code is constructed 
by moving from the root along three branches---marking a 1, 0 and 1, respectively---to 
reach the leaf that corresponds to 0.13. The E receives a shorter code than 
a Q, a letter that occurs less frequently. By systematically employing codes 
of varying length, Huffman's idea may reduce by a half or even more the number 
of code symbols that would be needed if the codes were of a fixed length.

Huffman did not invent the idea of a coding tree. His insight was that by 
assigning the probabilities of the longest codes first and then proceeding 
along the branches of the tree toward the root, he could arrive at an optimal 
solution every time. Fano and Shannon had tried to work the problem in the 
opposite direction, from the root to the leaves, a less efficient solution. 
When presented with his student's discovery, Huffman recalls, Fano exclaimed 
in his thick Italian accent: ``Is that all there is to it!''

Products that use Huffman code might fill a consumer electronics store. A 
recent entry on the shop shelf is VCR Plus+, a device that automatically 
programs a VCR and is making its inventors wealthy. (Some newspapers on their 
own list a toll-free number that readers can call for information about where 
to buy the device.) Instead of confronting the frustrating process of programming 
a VCR, the user simply types into the small handheld device a numerical code 
that is printed in the television listings. When it is time to record, the 
gadget beams its decoded instructions to the VCR and cable box with an infrared 
beam like those on standard remote-control devices. This turns on the VCR, 
sets it (and the cable box) to the proper channel and records for the designated time.

Although he acknowledges that he is best known for his code, Huffman says 
he is most proud of his doctoral thesis, which may be the first formal methodology 
for devising asynchronous sequential switching circuits, an important type 
of computer logic. The thesis helped him obtain a faculty position at M.I.T. 
to teach a course on switching circuits.

His work also attracted the attention of others. During the early 1960s, 
William O. Baker, then vice president of research for AT\&T Bell Laboratories, 
tapped Huffman to sit on a committee that was reviewing future technology 
plans for the National Security Agency. What may have attracted Baker was 
work by Huffman that had outlined a method for converting one sequence of 
binary numbers into another without losing any information in the translation, 
a technique that had obvious application in cryptography.

In 1967 Huffman left his position as full professor at M.I.T. to move to 
the University of California at Santa Cruz, which had lured him to become 
the first head of its new department of computer science. The relocation 
brought him closer to the western mountains where he loves to backpack and 
camp. (At the age of 65, he now prefers snorkeling and body surfing.) Today 
Huffman is no longer head of the department, but he still teaches a course 
in digital signal processing at the university.

Huffman's earliest years did not mark him as a prodigy. His mother once told 
him that he lagged behind other children by two years in learning how to 
speak. He attributes his slow development to a number of family incidents 
that led to his parents' divorce and that he has ever since tried to forget. 
His mother, whom he recalls with great affection, tried to help by becoming 
a mathematics teacher at a school for troubled children so he could be enrolled 
there. But a series of tests immediately made clear to his mother and teachers 
that his reticence had masked precociousness.

At school, Huffman soon leapfrogged his classmates. He finished a bachelor's 
degree in electrical engineering at Ohio State University at the age of 18 
and immediately became an officer in the U.S. Navy, where he served on a 
destroyer that helped to clear mines in Japanese and Chinese waters after 
World War II.

Huffman believes his tumultuous early years fostered a love of mathematics. 
``I like things neat,'' he says. ``I like to wrap things up and get definitive 
answers, possibly because of the uncertainties of my early life.'' A sense 
of order is something toward which he continues to strive. Huffman told this 
caller that he could spare only 20 minutes. When the time elapsed, an alarm 
dutifully sounded in the background.

The imposition of structure where none exists has proved a recurrent theme 
of his career. In the early 1970s Huffman became a debunker of optical illusions. 
What inspired him were the seemingly incongruous shapes in the work of M. 
C. Escher: triangles containing three right angles, for example. Inspecting 
Escher's creations, which he much admires, led him to devise a set of rules 
to determine whether an artist's picture or a video image had cheated in 
depicting a two-dimensional representation of a three-dimensional scene.

Huffman determined a method for showing whether the many boundaries between 
geometric elements in an image---represented as Y, V or T shapes, among others---logically 
fit into a coherent pattern. He describes his proof as an image grammar. 
``I wanted to create a sieve so grammatical pictures would go through and 
ungrammatical images would be seen as unrealizable,'' he says. This contribution 
to the young field of scene analysis, which has been used in developing machine 
vision systems for robots, was presented in a 1971 paper entitled ``Impossible 
Objects as Nonsense Sentences.''

Huffman's other work has ranged from the design of radar waveforms to his 
last paper, published in the early 1980s, which proved that a digital computer 
could be designed that would virtually eliminate one of the staples of Boolean 
algebra. Huffman showed that a hypothetical machine could function using 
only one NOT operation.

This logic element from Boolean algebra takes a zero or a one and converts 
it to its binary opposite (NOT zero but one). Huffman called a lecture he 
gave on the subject, ``How to Say No Once and Really Mean It.'' Says Huffman: 
``It was totally impractical, but it was a kind of a mind exercise that showed 
how it could be done. I enjoy pushing things to their theoretical limits.''

Since that time, Huffman has exchanged paper writing for paper folding. He 
wanted to see how the lines and intersections on the flat surfaces that he 
had pored over in his work on scene analysis could be folded into three-dimensional 
structures. Using a stylus to emboss lines into paper or thin vinyl sheets, 
he has concocted spirals, domes and other shapes. Huffman has lectured on 
the theory and practice of paper folding at M.I.T. and Stanford, among other 
institutions.

Paper folding goes along with a number of other whimsical pursuits. Huffman 
learned how to ride a unicycle from Claude Shannon and still keeps one in 
his garage. His living room, which is adorned with his contorted paper creations, 
is also sometimes graced with a large red circus ball and his invention of 
a Bongo Board that rolls in two axes. On Huffman's board, a rider stands 
atop a bowling ball rather than the standard cylindrical roller.

Although others have used Huffman code to help make millions of dollars, 
Huffman's main compensation was dispensation from a final exam. He never 
tried to patent an invention from his work and experiences only a twinge 
of regret at not having used his creation to make himself rich. ``If I had 
the best of both worlds, I would have had recognition as a scientist, and 
I would have gotten monetary rewards,'' he says. ``I guess I got one and 
not the other.''

If Huffman were just starting his career, patent attorneys would surely be 
knocking on his door. Patenting of algorithms is still subject to endless 
judicial debate. But a lawyer today would tell Huffman to ``clothe'' his 
code in silicon, that is, produce a patentable microchip that contains his 
code programmed into memory. ``I bet I could write an application that would 
be considered patentable,'' says Richard H. Stern, a patent attorney who 
was chief of the intellectual property section of the U.S. Department of 
Justice from 1970 to 1978.

But Huffman has received other compensation. Textbooks on data communications 
and other digital arts include sections on Huffman code. Huffman has received 
several awards from the Institute of Electrical and Electronics Engineers. 
And a few years ago an acquaintance told him that he had noticed that a reference 
to the code was spelled with a lowercase ``H.'' Remarked his friend to Huffman, 
``David, I guess your name has finally entered the language.''

\noindent---Gary Stix
% .tex must be omitted
\noindent The profile of David Huffman comes from~\cite{stix91}

% Pleasse do not use \nocite{*}.  It causes all the items in the bibliography to 
% be cited.
%\nocite{*}
\bibliographystyle{plainnat}
\bibliography{jitteredhuffmancode}  % This comes from the file: huffmanwithnoise.bib
% To check if your bibliography is correct run the command:
% bibtex huffmanwithnoise
% If this runs without any error messages your bibliography has no errors.
\vfill
\pagebreak
% \tableofcontents   % Not necessary for a small document, or an article
\end{document}
